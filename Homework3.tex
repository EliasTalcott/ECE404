\documentclass[11pt]{article}


%%% Packages
%%
\usepackage{amsmath}
\usepackage{amsfonts}
\usepackage{amssymb}
\usepackage{fancyhdr}
\usepackage{float}
\usepackage{graphicx}
\usepackage{listings}
\usepackage{enumitem}
\usepackage[margin = 1in, headheight = 13.6pt]{geometry}
\usepackage[linktoc=all]{hyperref}
%%
%%%


%%% Formatting
%%
\parindent 0em
\parskip 1em
\pagestyle{fancy}
\fancyhead{}
\fancyfoot{}
\fancyhead[L]{\slshape\MakeUppercase{{\myTitle}}}
\fancyhead[R]{\slshape{\myName}}
\fancyfoot[C]{\thepage}
%%
%%%


%%% User defined variables
%%
\def \myTitle {ECE 404 Homework 3}
\def \myName {Elias Talcott}
\def \myDate {February 6, 2020}
%%
%%%


\begin{document}

\begin{titlepage}
\title{\myTitle}
\author{\myName}
\date{\myDate}
\maketitle
\vspace{1in}
\tableofcontents
\thispagestyle{empty}
\end{titlepage}

\section{Theory Problems}

%%
% Problem 1
\subsection{Problem 1}
Show whether or not the set of remainders $Z_{12}$ forms a group with either one of the modulo addition or modulo multiplication operations.

\textbf{Solution}

%%
% Problem 2
\subsection{Problem 2}
Compute $gcd(29495, 16983)$ using Euclid's Algorithm. Show all the steps.

\textbf{Solution}

%%
% Problem 3
\subsection{Problem 3}
With the help of Bezout's identity, show that if c is a common divisor of two integers $a, b > 0$, then $c\; |\; gcd(a,b)$ (i.e. c is a divisor of $gcd(a,b)$)

\textbf{Solution}

%%
% Problem 4
\subsection{Problem 4}
Use the Extended Euclid's Algorithm to compute by hand the multiplicative inverse of 25 in $Z_{}28$. List all of the steps.

\textbf{Solution}

%%
% Problem 5
\subsection{Problem 5}
In the following, find the smallest possible integer x. Briefly explain how you found the answer to each. You should solve them without using brute-force methods.
\begin{enumerate}[label=(\alph*)]
\item $8x \equiv 11\; (\textrm{mod } 13)$
\item $5x \equiv 3\; (\textrm{mod } 21)$
\item $8x \equiv 9\; (\textrm{mod } 7)$
\end{enumerate}

\textbf{Solution}
\begin{enumerate}[label=(\alph*)]
\item 
\item 
\item 
\end{enumerate}

\pagebreak

\section{Programming Problem}

Write a program that takes as input a small integer n (say, smaller than 50) and determines if Zn is a field or only a commutative ring. Assume that the operators are modulo n addition and modulo n multiplication. The program should prompt the user to enter the number. Depending upon the input n, it should correctly print out either ``field'' or ``ring''.

\subsection{Python Code}
\lstinputlisting[language = Python]{Fields.py}
\pagebreak

\subsection{Code Explanation}
If an integer n is prime, then its set of residues $Z_n$ along with the addition and multiplication operators form a finite field. This is the case because a prime modulus is relatively prime with each member of its set of residues. If a number is relatively prime with the modulus, then it has a multiplicative inverse (a requirement for a ring to become a field).

My solution simply checks whether the modulus is prime or not. If it is prime, the program prints ``field''. If it is not prime, the program prints ``ring''.

\end{document}